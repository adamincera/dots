\documentclass{article}
\usepackage[utf8]{inputenc}
\usepackage[T1]{fontenc} % ASCII-friendly formatting of text
\usepackage{parskip} % no indentation of paragraphs
\usepackage{courier} % for use in lstlistings
\usepackage[margin=1in]{geometry}

\usepackage{titlesec} % for making sub-subsections
\setcounter{secnumdepth}{3}

\usepackage{float} % for manipulating floats like tables and figures
\restylefloat{table} % allows you to use "H" for hard-fixing table position
\usepackage{graphicx} % for including figures

\usepackage{csquotes} % for block quotes

\usepackage{hyperref}
\newcommand\fnurl[2]{%
  \href{#2}{#1}\footnote{\url{#2}}%
} %for putting hyperlinks in footnotes

\usepackage{listings} % include source code
\usepackage{color} % color source code

\newcommand{\pltlang}{d.o.t.s.} % allows us to just reference the language name with the ``\pltlang'' cmd, so we can change the name later w/o too much hassle

\newcommand{\code}[1]{\texttt{#1}} %sets it up so you can use \code{...} to format text

\definecolor{codegreen}{rgb}{0,0.6,0}
\definecolor{codegray}{rgb}{0.5,0.5,0.5}
\definecolor{codebrown}{rgb}{0.7,0.35,0.25}
\definecolor{backcolor}{rgb}{0.95,0.95,0.92}

\lstdefinestyle{pltStyle}{
  basicstyle=\ttfamily,
  backgroundcolor=\color{backcolor},
  commentstyle=\color{codebrown},
  keywordstyle=\color{blue},
  numberstyle=\tiny\color{codegray},
  stringstyle=\color{codegreen},
  breakatwhitespace=false,
  breaklines=true,
  captionpos=b,
  keepspaces=false,
  numbers=left,
  numbersep=5pt,
  showspaces=false,
  showstringspaces=false,
  showtabs=false,
  tabsize=2
}

\lstdefinelanguage{pltLang}
{
  morekeywords={
    bool,
    true,
    false,
    num,
    string,
    null,
    INF,
    if,
    else,
    for,
    while,
    break,
    in,
    node,
    graph,
    list,
    dict,
    print,
    range,
    def,
    return
  },
  sensitive=true, %keywords are case sensitive
  morecomment=[l]{\#}, % symbol for single-line comment
  morecomment=[s]{\\*}{*\\}, % symbol for multi-line comment
  morestring=[b]" % sets double quote as string indicator
}

\lstset{style=pltStyle, columns=flexible}

\title{d.o.t.s. \\ A graph language.}
\author{Hosanna Fuller (hjf2106) --- Manager\\
Rachel Gordon (rcg2130) --- Language Guru\\
Yumeng Liao (yl2908) --- Tester\\
Adam Incera (aji2112) --- System Architect}
\date{September 2015}

\begin{document}

\maketitle

\tableofcontents
\newpage

\section{Lexical Elements}

\section{Data Types}

\subsection{Primitive Types}

\subsubsection{num}

The \code{num} data type represents all numbers in \pltlang\ There is no distinction between the traditional data types of \code{int} and \code{float}, which means for example that there is no difference between the values \code{5} and \code{5.0}. The comparative ordering of nums is the same as that of numbers in mathematics. 

\begin{lstlisting}[language=pltLang, caption=Declaration of ``num'' types., label=lst:num]
num x = 5;
num y = 5.0;
num z = x;

num q = 3.14159;

num a, b, c;
\end{lstlisting}

In Listing \ref{lst:num} variables \code{a, b, c, x, y, z}, and \code{q} are all of the type \code{num}. Variables \code{x}, \code{y}, and \code{z} store equivalent values. Variables \code{a}, \code{b}, and \code{c} are all equal to \code{null}.

\subsubsection{string}

A \code{string} is a sequence of 0 or more characters. Comparative ordering of strings is determined sequentially by comparing the ASCII value of each character in the two strings from left to right.

\begin{lstlisting}[language=pltLang, caption=Declaration of ``string'' types., label=lst:string]
string a = "alpha";
string empty = "";
string char = "a";
\end{lstlisting}

\subsubsection{bool}

The \code{bool} type is a logical value which can be either the primitive values \code{true} or \code{false}.

\begin{lstlisting}[language=pltLang, caption=Declaration of ``bool'' types., label=lst:bool]
bool t = true;
bool f = false;
\end{lstlisting}

\section{Expressions and Operators}

\section{Statements}

\section{Functions}

\subsection{Function Declaration and Definition}

Before a function can be used, it must be declared and defined. Functions are declared using the \code{def} keyword, followed by the data type the function will return, followed by the function name, followed by a list of parameters enclosed in parentheses. The function must then be immediately defined within a set of curly braces immediately follwing the parentheses of the parameter list. 

\begin{lstlisting}[language=pltLang, caption=Function declaration and definition., label=lst:funct-def]
\* 
 * Outline of function declaration and definition.
 * ``return_type'' would be a data type.
 *\
def return_type function_name () {
  \* function implementation code *\
}
\end{lstlisting}

\subsection{Return Statements}

Each function must return a value that matches the declared return type using the \code{return} keyword. For functions with the \code{null} return type, indicating that nothing is returned by the function, the return statement can consist either of the keyword \code{return} as an expression by itself (line 2 of Listing \ref{lst:funct-return}), or it can explcitly \code{return null} (line 6 of Listing \ref{lst:funct-return}).

\begin{lstlisting}[language=pltLang, caption=Return statements of functions., label=lst:funct-return]
def null fnull1 () {
  return;
}

def null fnull2 () {
  return null;
}

def int fint () {
  return 4;
}
\end{lstlisting}

\subsection{Parameter List}

The declaration of a function must include a list of required parameters enclosed within parentheses. To define a function which requires no parameters, the contents of the parentheses can be left blank. Otherwise, each parameter requires the data type, followed by a variable name by which the parameter can be referenced within the function definition. 

\begin{lstlisting}[language=pltLang, caption=Parameters in function declarations., label=lst:funct-params]
def null no_params () {
  return;
}

def num one_param (num x) {
  num b = x;
  return b;
}

def string multi_params (string s1, num y, string s2) {
  string statement = s1 + " " + " " + y + "s2";
  return statement;
}

\end{lstlisting}

\subsection{Calling Functions}

The syntax for calling a function is: the name of the function, followed by a comma-separated list of values or variables to be used in paremter list enclosed within parentheses. Each value or variable passed in to a function call is mapped to the corresponding variable in the declared parameter list of the function.

A function-call expression is considered of the same type as its return type. Because of this, function-call expressions may be used as any other expression. For example a function-call expression can be used in the assignment of variables, as in line 11 of Listing \ref{lst:funct-call}.

\begin{lstlisting}[language=pltLang, caption=Function declaration and definition., label=lst:funct-call]
def num increment (num n, num incr) {
  return n + incr;
}

num x = 4;

\* The following call maps ``x'' to the variable ``n'',
 * and ``2'' to the variable ``incr'' from the declaration
 * of the ``increment'' function 
 *\
num y = increment(x, 2);

print("y: ", y); # prints --> ``y: 6''
\end{lstlisting}

\subsection{Variable Length Parameter Lists}

The \emph{only} function in \pltlang\ that can have a variable number of parameters is the built-in \code{print} function. All other functions must be declared with a defined absolute number of 0 or more parameters. 

The \code{print} function may be called using a comma-separated list of expressions which can be evaluated as or converted to the \code{string} type. Each of the built-in types may be used directly as an argument to the print function.

\begin{lstlisting}[language=pltLang, caption=The built-in ``print'' function., label=lst:funct-print]
string alpha = "World";
print("Hello", alpha, "\n");

node x("foo");
num n = 20;
print("The node <", x, "> has an associated num equal to:", n, "\n");
\end{lstlisting}

In Listing \ref{lst:funct-print}, the \code{print} function was called on line 2 with 3 arguments and with 5 arguments on line 6. The number of arguments passed to \code{print} does not matter. 

\section{Program Structure and Scope}

\subsection{Program Structure}

A \pltlang\ program consists of a series of function declarations and expressions. Because \pltlang\ is a scripting language, there is no \code{main} function. Instead, expressions are executed in order from top to bottom. Functions must be declared and defined before use. 

\subsection{Scope}

\section{Sample Program}

\end{document}
